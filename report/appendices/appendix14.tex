%%% Appendix A
\chapter{UVM classes}
\label{appendix14}

	\lstinputlisting[style=sv,language=Verilog, breaklines=true]{../hardware/dlx/test_bench/uvm_class_def/dlx_sequence_item.sv}


	\lstinputlisting[style=sv,language=Verilog, breaklines=true]{../hardware/dlx/test_bench/uvm_class_def/dlx_sequence.sv}
	
	
	\lstinputlisting[style=sv,language=Verilog, breaklines=true]{../hardware/dlx/test_bench/uvm_class_def/dlx_sequencer.sv}
	
	
	\lstinputlisting[style=sv,language=Verilog, breaklines=true]{../hardware/dlx/test_bench/uvm_class_def/dlx_driver.sv}
	
	
	\lstinputlisting[style=sv,language=Verilog, breaklines=true]{../hardware/dlx/test_bench/uvm_class_def/dlx_monitor.sv}
	
	
	\lstinputlisting[style=sv,language=Verilog, breaklines=true]{../hardware/dlx/test_bench/uvm_class_def/dlx_env.sv}
	
		\lstinputlisting[style=sv,language=Verilog, breaklines=true]{../hardware/dlx/test_bench/uvm_class_def/dlx_scoreboard.sv}
		
				\lstinputlisting[style=sv,language=Verilog, breaklines=true]{../hardware/dlx/test_bench/uvm_class_def/dlx_test.sv}
% \lstinputlisting is an alternative way to import text or code from an external file. In this example the behavioural VHDL description of an adder contained in the file adder.vhd is imported. 
% Note that you can set the language of the code that you want to import (VHDL in this example). When you set the language you will see the keywords of that specific language highlighted in your output pdf file.
%You can set a lot parameters: for some examples take a look at the chapter 'How to document the project' that can you find in DLX_Project.pdf.